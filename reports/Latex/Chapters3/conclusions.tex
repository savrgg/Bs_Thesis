\chapter{Conclusiones}
\label{ch:conclusion}

% Párrafo intro 
En este trabajo se expone la técnica de optimización de Newton-Lanczos aplicada al problema de aprendizaje supervisado del Análisis Discriminante Lineal de Fisher (ADLF). El ADLF busca encontrar la mejor matriz de proyección que maximice el cociente de trazas, logrando que los individuos de una misma clase sean proyectados lo más cercano posible entre ellos y lo más separado posible de otra clase. Anteriormente, la solución era considera computacionalmente costosa de resolver; sin embargo, en la actualidad se puede aprovechar la velocidad de convergencia del método de Newton y la rapidez del algoritmo de Lanczos para calcular eigenpares. 

% ------ primero

El método de Lanczos es muy efectivo para calcular solamente algunos de los eigenpares de las matrices. Bajo el contexto de aritmética inexacta, Lanczos implementado con reortogonalización completa resulta ser más eficiente que la factorización SVD, cuando se desea calcular menos del 5 \% de los eigenpares. Aunque este número parece ser demasiado pequeño resulta de gran utilidad; por ejemplo, si la matriz tiene dimensión de $1000 \times 1000$, tener 50 eigenpares podría ser adecuado para el problema. 

\newpage
En este trabajo de tesis se comparó el método ADLF vía Newton-Lanczos con el Análisis Discriminante Lineal (ADL) y la técnica de Regresión Lineal Múltiple (RLM). La comparación se realizó empleando las bases \textit{State Farm} y \textit{Otto Group}. En el primero, se logró una precisión mejor que los otros dos métodos con un tiempo de cómputo similar. Para el segundo, la precisión entre los tres métodos es muy similar con tiempos de cómputo similares.

El método funciona muy bien para problemas de mediana y gran dimensionalidad. Si se trata de imágenes es muy importante que tengan un preprocesamiento adecuado, ya que si el objeto a clasificar se encuentra en distintos píxeles de la imagen, no se tendrá buenos resultados.

Las principales conclusiones, aportaciones y aprendizajes de esta tesis son las siguientes:

\begin{itemize}
\item Se implementó computacionalmente una técnica de optimización que anteriormente resultaba difícil de resolver.
\item El ADLF es un método muy flexible, ya que no requiere ningún supuesto sobre la distribución de los datos.
\item La eficiencia del ADLF es superior a la RLM y similar al ADL. Los resultados fueron satisfactorios.
\item La precisión del método resulta comparable con los métodos de clasificación lineal, llegando incluso a mejorarlos.
\item El ADLF permite visualizar los datos en dimensiones menores separando las clases lo mejor posible.
\end{itemize}

Una de las complicaciones del algoritmo de Lanczos es la reortogonalización de la base. En esta tesis se utilizó el método de reortogonalización completa; sin embargo, existen modificaciones al algoritmo que pueden ser exploradas con el objetivo de lograr mayor eficiencia en términos computacionales. Por ejemplo, J.W. Demmel (1997) \cite{demmel1997applied} propone algunas alternativas que pueden ser utilizadas para mejorar el proceso de reortogonalización de la base en el algoritmo de Lanczos.



