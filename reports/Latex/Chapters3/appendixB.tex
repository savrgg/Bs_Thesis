\chapter{Apéndice B: Códigos en R}
\label{ch:appendixB}

\section{Experimentos numéricos}
\subsection{$01\_Prueba20comp$}

\begin{lstlisting}[language=R, basicstyle=\small]
library(ProjectTemplate)
reload.project()

# Jaffe.20 & Mnist.20  are 5 element lists:
# 1) iterative model
# 2) data frame with \rho, f\rho values
# 3) projection matrix V
# 4) test error
# 5) knn model

Jaffe.20 <- prueba.20comp(JAFFE.train, JAFFE.test, "JAFFE")
Mnist.20 <- prueba.20comp(MNIST.train, MNIST.test, "MNIST")

Jaffe.20$modelo.iterativo
Mnist.20$modelo.iterativo
\end{lstlisting}

\subsection{$02\_Models\_comparisons$}
\begin{lstlisting}[language=R, basicstyle=\small]
# 1) Data y proyeccion size --------------------------------
library(ProjectTemplate)
reload.project()

numComp.Jaffe <- c(10,13,16,19,22,25,28,31,
                   34,37,40,43,46,49) 
numComp.Mnist <- c(2,5,10,15,20,40,60,80,
                   100,120,140,160,180,193)

comp.error.MNIST<-comparacion.modelos(train.p = MNIST.train,
                                 test.p = MNIST.test, 
                                 dataset = "MNIST", 
                                 numComp = numComp.Mnist,
                                 fortran = FALSE)

comp.error.JAFFE<-comparacion.modelos(train.p = JAFFE.train,
                                  test.p = JAFFE.test, 
                                  dataset = "JAFFE", 
                                  numComp = numComp.Jaffe,
                                  fortran = FALSE)
# 2) Graphs ------------------------------------------------

graficar.comparacion(datos = comp.error.MNIST, 
                     numComp = numComp.Mnist, 
                     label = "MNIST")

graficar.comparacion(datos = comp.error.JAFFE, 
                     numComp = numComp.Jaffe, 
                     label = "JAFFE")
\end{lstlisting}

\subsection{$03\_Profiling$}
\begin{lstlisting}[language=R, basicstyle=\small]
# 1) Data y proyeccion size --------------------------------
library(ProjectTemplate)
reload.project()

# 1) Componentes ------------------------------------------
numComp.Jaffe <- c(10,13,16,19,22,25,28,31,
                   34,37,40,43,46,49) 
numComp.Mnist <- c(2,5,10,15,20,40,60,80,
                   100,120,140,160,180,193)
# 2) Profiling ----------------------------------------
prof.Mnist <- prolifiling.model(train.p = MNIST.train, 
                                test.p = MNIST.test,
                                dataset = "MNIST",
                                numComp = numComp.Mnist)
prof.Jaffe <- prolifiling.model(train.p = JAFFE.train, 
                                test.p = JAFFE.test,
                                dataset = "JAFFE",
                                numComp = numComp.Jaffe)

# 3) Graphs -------------------------------------------
graficar.profiling(times.profiling = prof.Mnist, 
                   numComp = numComp.Mnist, 
                   dataset = "MNIST")
graficar.profiling(times.profiling = prof.Jaffe, 
                   numComp = numComp.Jaffe, 
                   dataset = "JAFFE")
\end{lstlisting}

\section{Funciones}
\subsection{$load\_libraries$}
\begin{lstlisting}[language=R, basicstyle=\small]
# 1) Load libraries ----------------------------------------
set.seed(12345)
library(plyr) # Tools for split, applying and combine data
library(dplyr) # package of grammar of data manipulation
library(tidyr) # package to efficiently "tidy" data
library(FactoMineR) # Package for multivariate analysis
library(ggplot2) # Graphing package "Grammar of Graphics"
library(gridExtra) # Package to work with grid graphics
library(class) # Functions for classification, including knn
library(nnet) # Software for multinomial linear models
library(MASS) # "Applied Statistics with S"



# 1) Fisher eficiente ---------------------------------
lda.iter.ef <- function (test.p, 
                         train.p, 
                         espacio.proy,
                         prior = proportions,
                         tol = 1e-10) {
  
  # Prepare matrix of data
  x <- train.p %>% dplyr::select(-grupo)
  x <- as.matrix(x)
  
  # Get dimensions
  n <- nrow(x)
  p <- ncol(x)
  
  # Get number of groups and the levels
  g <- as.factor(train.p$grupo)
  lev <- lev1 <- levels(g)
  
  # Get the number of images per class
  counts <- as.vector(table(g))
  proportions <- counts/n
  ng <- length(proportions)
  names(prior) <- names(counts) <- lev1
  
  # Group means and total mean
  group.means <- tapply(c(x), list(rep(g, p), col(x)), mean)
  all.means <- as.matrix(colMeans(x)) %>% t()
  
  # Scatter Matrices 
  #Intra class
  B <- t(x - group.means[g, ]) %*% (x - group.means[g, ]) 
  # Between class
  A <- (t(group.means)-matrix(rep(all.means,10),ncol=10))%*% 
    diag(counts) %*% 
    t(t(group.means)-matrix(rep(all.means,10),ncol=10)) 
  
  # Initialization
  rhoAnt = -1000000
  iter=1
  dimension = espacio.proy
  # rho = (cota.inf + cota.sup)/2
  rho = 0
  frho = sum(eigen(A-rho*B)$values[1:dimension])
  V = eigen(A-rhoAnt*B)$vectors[,(1:dimension)]
  
  # list of results
  a=list()
  a[[1]] <- list(iter = 1, rho = rho, frho = frho, V = V) 
  
  # Iterative method (use of efficient irlba method)
  while(iter < 50 & abs(rho-rhoAnt)>tol){
    print(paste("iteracion: ",as.character(iter)))
    rhoAnt = rho
    iter=iter+1  
    V = eigen(A-rhoAnt*B)$vectors[,(1:dimension)] # Lanczos
    rho = sum(diag(t(V) %*% A %*% V)) /
    sum(diag(t(V) %*% B %*% V))
    frho = sum(eigen(A-rho*B)$values[1:dimension])
    a[[iter]] <- list(iter = iter, rho = rho,
                      frho = frho, V = V)
  }
  
  # Getting the projection matrix
  scaling <- V
  dimnames(group.means)[[2L]] <- colnames(x)
  cl <- match.call()
  cl[[1L]] <- as.name("lda.iter")
  structure(list(iteraciones = a, 
                 prior = prior, 
                 counts = counts, 
                 means = group.means, 
                 scaling = scaling, 
                 lev = lev, 
                 N = n, 
                 call = cl, 
                 class = "lda.iter"))
}

# 2) Fisher para pruebas ------------------------------
lda.iter <- function (test.p, 
                      train.p, 
                      espacio.proy,
                      prior = proportions, 
                      tol = 1e-10) {
  
  # Prepare matrix of data
  x <- train.p %>% dplyr::select(-grupo) %>% as.matrix()
  
  # Get dimensions
  n <- nrow(x)
  p <- ncol(x)
  
  # Get number of groups and the levels
  g <- as.factor(train.p$grupo)
  lev <- lev1 <- levels(g)
  
  # Get the number of images per class
  counts <- as.vector(table(g))
  proportions <- counts/n
  ng <- length(proportions)
  names(prior) <- names(counts) <- lev1
  
  # Group means and total mean
  group.means <- tapply(c(x), list(rep(g, p), col(x)), mean)
  all.means <- as.matrix(colMeans(x)) %>% t()
  
  # Scatter Matrices 
  #Intra class
  B <- t(x - group.means[g, ]) %*% (x - group.means[g, ]) 
  # Between class
  A <- (t(group.means)-matrix(rep(all.means,10),ncol=10))%*% 
    diag(counts) %*% 
    t(t(group.means)-matrix(rep(all.means,10),ncol=10)) 
  
  # [not run] Intra class + Between class = Total variance
  diag(A+B) 
  diag(49*(as.matrix(x) %>% cov))
  
  # [not run] EigenValues Time
  (eigen(B))$values
  (eigen(A))$values
  
  # [not run] Boundaries
  cota.inf <- sum((eigen(A))$values[1:espacio.proy])/
    sum((eigen(B))$values[1:espacio.proy])
  cota.sup <- sum((eigen(A))$values[1:espacio.proy])/
sum((eigen(B))$values[(dim(A)[1]+1-espacio.proy):dim(A)[1]])
  
  # Initialization
  rhoAnt = -1000000
  iter=1
  dimension = espacio.proy
  rho = (cota.inf + cota.sup)/2
  
  frho = sum(eigen(A-rho*B)$values[1:dimension])
  V = eigen(A-rhoAnt*B)$vectors[,(1:dimension)]
  
  # list of results
  a=list()
  a[[1]] <- list(iter = 1, rho = rho, frho = frho, V = V) 
  
  # Iterative method (use of efficient irlba method)  
  while(iter < 50 & abs(rho-rhoAnt)>tol){
    print(paste("iteracion: ",as.character(iter)))
    rhoAnt = rho
    iter=iter+1  
    V = eigen(A-rhoAnt*B)$vectors[,(1:dimension)]
rho = sum(diag(t(V)%*%A %*% V)) /sum(diag(t(V) %*% B %*% V))
    frho = sum(eigen(A-rho*B)$values[1:dimension])
a[[iter]] <-list(iter = iter, rho = rho, frho = frho, V = V)
  }
  
  # Getting the projection matrix
  scaling <- V
  dimnames(group.means)[[2L]] <- colnames(x)
  cl <- match.call()
  cl[[1L]] <- as.name("lda.iter")
  structure(list(iteraciones = a, 
                 prior = prior, 
                 counts = counts, 
                 means = group.means, 
                 scaling = scaling, 
                 lev = lev, 
                 N = n, 
                 call = cl, 
                 cota.inf = cota.inf, 
                 cota.sup = cota.sup, 
                 class = "lda.iter"))
}

# 3) LDA original -------------------------------------
lda.original <- function (x, 
                          grouping, 
                          prior = proportions, 
                          tol = 1e-04, 
                          nu = 5) {
  
  # Prepare matrix of data
  x <- as.matrix(x)
  n <- nrow(x)
  
  # Get dimensions
  p <- ncol(x)
  g <- as.factor(grouping)
  
  # Get number of groups and the levels
  lev <- lev1 <- levels(g)
  counts <- as.vector(table(g))
  
  # Get the number of images per class
  proportions <- counts/n
  ng <- length(proportions)
  names(prior) <- names(counts) <- lev1
  
  # Group means 
  group.means <- tapply(c(x), list(rep(g, p), col(x)), mean)
  f1 <- sqrt(diag(var(x - group.means[g, ])))
  scaling <- diag(1/f1, , p)
  
  # Calculate and project
  fac <-  1/(n - ng)
  X <- sqrt(fac) * (x - group.means[g, ]) %*% scaling
  X.s <- svd(X, nu = 0L)
  rank <- sum(X.s$d > tol)
  scaling <- scaling %*% X.s$v[, 1L:rank] %*% 
    diag(1/X.s$d[1L:rank], , rank)
  xbar <- colSums(prior %*% group.means)
  fac <-1/(ng - 1)
  X <- sqrt((n * prior) * fac) * scale(group.means, 
        center = xbar, scale = FALSE) %*% scaling
  X.s <- svd(X, nu = 0L)
  rank <- sum(X.s$d > tol * X.s$d[1L])
  scaling <- scaling %*% X.s$v[, 1L:rank]
  
  # Export
  dimnames(scaling) <- list(colnames(x), paste("LD",
          1L:rank, sep = ""))
  dimnames(group.means)[[2L]] <- colnames(x)
  cl <- match.call()
  cl[[1L]] <- as.name("lda")
  structure(list(prior = prior, counts = counts, 
                 means = group.means, 
                 scaling = scaling, lev = lev, 
                 svd = X.s$d[1L:rank], N = n, 
                 call = cl), class = "lda")
}

# 4) Lanczos Fortran -----------------------------------
# Give r0, A
lanczos.iter <- function(A){
  Dim = dim(A)[2]
  r0 = matrix(rep(0,Dim), ncol = 1)
  for(i in 1:Dim){
    r0[i] = runif(1, 0, 1) 
  }
r0 = r0 / sqrt(sum(r0^2))
r = list()
b = list()
a = list()
q = list()

q[[1]] = 0 # q0
r[[1]] = r0 #r0
b[[1]] = sqrt(sum(r[[1]]^2)) # b1
q[[2]] = as.matrix(r[[1]]/b[[1]], ncol =D) #q1
a[[1]] = t(q[[2]]) %*% A %*% q[[2]] #a1
q[[1]]
j <- 2
q = cbind(q[[1]], q[[2]])
while(b[[j-1]] > .0000001 | j < Dim+1){
  z = A %*% q[,j]
  z = z - q %*% (t(q) %*% z)
  z = z - q %*% (t(q) %*% z)
  b[[j]] = sqrt(sum(z^2))
  q.temp = z/b[[j]]
  q <- cbind(q, q.temp)
  a[[j]] = t(q[,j+1]) %*% A %*% q[,j+1]
  j = j+1
  print(j)
}

dyn.load("/usr/lib/liblapack.dylib")

Jobz <- as.character('V')#Get The eigenval and Eigenvector
N <- as.integer(Dim)#Matrix Dimension
D <- unlist(a)[1:Dim]#Diagonal
E <- unlist(b)[2:(Dim)]#Upper (and lower) diagonal
Z <- array(0.0,N*N)#storage for eigenvectors
Ldz <- N#Leading dimension of Z (in our case number of rows)
Work <- array(0.0,2*N)#Storage for processing
info <- as.integer(0)#Info flag

res<-.Fortran("dstev", Jobz, N, D, E, Z, Ldz, Work, info)

evalues <- res[[3]]
evectors <- matrix(res[[5]],N,N)
list(evalues, evectors)
}

# 5) Fisher eficiente ---------------------------------
lda.iter.fortran<- function (test.p = test.p, 
                         train.p = train.p, 
                         espacio.proy,
                         prior = proportions,
                         tol = 1e-10) {
  
  # Prepare matrix of data
  x <- train.p %>% dplyr::select(-grupo)
  x <- as.matrix(x)
  
  # Get dimensions
  n <- nrow(x)
  p <- ncol(x)
  
  # Get number of groups and the levels
  g <- as.factor(train.p$grupo)
  lev <- lev1 <- levels(g)
  
  # Get the number of images per class
  counts <- as.vector(table(g))
  proportions <- counts/n
  ng <- length(proportions)
  names(prior) <- names(counts) <- lev1
  
  # Group means and total mean
  group.means <- tapply(c(x), list(rep(g, p), col(x)), mean)
  all.means <- as.matrix(colMeans(x)) %>% t()
  
  # Scatter Matrices 
  #Intra class
  B <- t(x - group.means[g, ]) %*% (x - group.means[g, ]) 
  # Between class
  A <- (t(group.means)-matrix(rep(all.means,10),ncol=10))%*% 
    diag(counts) %*% 
    t(t(group.means)-matrix(rep(all.means,10),ncol=10)) 
  
  # Initialization
  rhoAnt = -1000000
  iter=1
  dimension = espacio.proy
  # rho = (cota.inf + cota.sup)/2
  rho = 0
  frho = sum(eigen(A-rho*B)$values[1:dimension])
  V = eigen(A-rhoAnt*B)$vectors[,(1:dimension)]
  
  # list of results
  a=list()
  a[[1]] <- list(iter = 1, rho = rho, frho = frho, V = V) 
  
  # Iterative method (use of efficient irlba method)
  while(iter < 50 & abs(rho-rhoAnt)>tol){
    print(paste("iteracion: ",as.character(iter)))
    rhoAnt = rho
    iter=iter+1  
    
  V = lanczos.iter(A-rhoAnt*B)[[2]][,(1:dimension)]#Lanczos
    rho = sum(diag(t(V) %*% A %*% V)) /
      sum(diag(t(V) %*% B %*% V))
    frho = sum(lanczos.iter(A-rhoAnt*B)[[1]][1:dimension])
    a[[iter]] <- list(iter = iter, rho = rho,
                      frho = frho, V = V)
  }
  
  # Getting the projection matrix
  scaling <- V
  dimnames(group.means)[[2L]] <- colnames(x)
  cl <- match.call()
  cl[[1L]] <- as.name("lda.iter")
  structure(list(iteraciones = a, 
                 prior = prior, 
                 counts = counts, 
                 means = group.means, 
                 scaling = scaling, 
                 lev = lev, 
                 N = n, 
                 call = cl, 
                 class = "lda.iter"))
}
\end{lstlisting}

\subsection{$functions\_comparison$}
\begin{lstlisting}[language=R, basicstyle=\small]
comparacion.modelos <- function(train.p, test.p, 
                                dataset,numComp, 
                                fortran = FALSE){
  # lda iterativo
  print(paste("===Corriendo el modelo con la base",dataset))
  
  lda.iter <- sapply(1:14, function(x){
    print(paste("Iteracion con: ", 
                numComp[x], 
                " componentes."))
    
    # Run model with different number of components
    if(fortran == FALSE){
    modelo.iterativo <- lda.iter.ef(train.p = train.p, 
                                    test.p = test.p, 
                                  espacio.proy = numComp[x])
    }else{
      modelo.iterativo <- lda.iter.fortran(train.p = train.p, 
                                      test.p = test.p, 
                                  espacio.proy = numComp[x])
    }
    # Data.frame with information of  (\rho and f(\rho))
    iter <- sapply(1:length(modelo.iterativo$iteraciones), 
                   function(x){
          data.frame(modelo.iterativo$iteraciones[[x]]$rho,
                  modelo.iterativo$iteraciones[[x]]$frho)%>% 
                setNames(c("rho", "frho")) %>% data.frame()
                   }) %>% t() %>% data.frame() %>% 
      mutate(iter = 1:nrow(.), 
             rho = unlist(rho),
             frho = unlist(frho))
    iter <- iter %>% data.frame() %>% 
      dplyr::mutate(frho2 = c(iter$frho[2:nrow(.)],NA ),
                    rho2 = c(iter$rho[2:nrow(.)], NA))
    
    # Projected data into a 20-dimensional space
    V <- modelo.iterativo$iteraciones[[
      length(modelo.iterativo$iteraciones)]]$V
    
    train.proyectados <- data.frame(as.matrix(train.p %>% 
                          dplyr::select(-grupo)) %*% V) %>%
      mutate(grupo = train.p$grupo)
    
    test.proyectados <- data.frame(as.matrix(test.p %>% 
                          dplyr::select(-grupo)) %*% V) %>%
      mutate(grupo = test.p$grupo)
    
    # K-nn model
    
    print("Iniciando modelo knn")
  modelo <- knn(train = train.proyectados[,c(1:numComp[x])], 
                test = test.proyectados[, c(1:numComp[x])], 
                  cl = train.proyectados$grupo, k = 3)
    pred <- data.frame(original = test.proyectados$grupo, 
                       predicho = modelo)
    
    pred <- pred %>% mutate(verdad = original != predicho)
    sum(pred$verdad)/nrow(pred)
  })
  
  modelo.logistico <- lapply(1:14, function(x){
    modelo <- multinom(grupo~., 
          data = train.p[,c(1:numComp[x], dim(train.p)[2])],
                       MaxNWts = 10000)
    predichos <- predict(modelo, 
                         newdata = test.p[,c(1:numComp[x])])
    pred <- data.frame(original = test.p$grupo, 
                       predicho = predichos)
    pred <- pred %>% mutate(verdad = original != predicho)
    sum(pred$verdad)/nrow(pred)
  })
  
  lda.original <- lapply(1:14, function(x){
    modelo<- lda(grupo ~ ., 
                 train.p[,c(1:numComp[x], 
                            dim(train.p)[2])], 
                 prior = c(1,1,1,1,1,1,1,1,1,1)/10)
  predichos <- predict(modelo, test.p[,1:numComp[x]])$class
    pred <- data.frame(original = test.p$grupo, 
                       predicho = predichos)
    pred <- pred %>% mutate(verdad = original != predicho)
    sum(pred$verdad)/nrow(pred)
  })
  
  data.frame(lda.iter = unlist(lda.iter), 
             logistico = unlist(modelo.logistico), 
             lda.original = unlist(lda.original))
}

graficar.comparacion <- function(datos, numComp, label){
  plot1 <- datos %>% 
    #write.table(sep = ';', pipe('pbcopy'), row.names= F)
    mutate(numComp = numComp) %>% 
    gather(variable, value, lda.iter:lda.original) %>% 
    ggplot(aes(x = numComp , y = 1-value, 
      group = factor(variable), color = factor(variable))) +
    geom_point() + geom_line()+
    labs(title = "Tasa de reconocimiento conforme 
         numero el espacio de proyeccion", 
         x = "numero de entrenamiento por grupo", 
         y="tasa reconocimiento")
  
  png(filename =paste(getwd(), "/graphs/Chapter4_comp_", 
        label, ".png", sep =""), width = 600,  height = 400)
  grid.arrange(plot1, ncol =1)
  dev.off()
}
\end{lstlisting}

\subsection{$functions\_prueba20comp$}
\begin{lstlisting}[language=R, basicstyle=\small]
# funciones prueba 20 componentes

plot.comp <- function(V1, label, train.p, test.p){
  train.proyectados <- data.frame(as.matrix(train.p %>% 
                          dplyr::select(-grupo)) %*% V1) %>%
    mutate(grupo = train.p$grupo)
  
  test.proyectados <- data.frame(as.matrix(test.p %>% 
                          dplyr::select(-grupo)) %*% V1) %>%
    mutate(grupo = test.p$grupo)
  
  a1 = ggplot(train.proyectados) + 
    geom_point(aes(x=X1, y=X2, 
                   group = grupo, color = grupo), 
               size=5, alpha = .2) + 
    geom_point(aes(x=X1, y=X2, 
                   group = grupo, color = grupo))+ 
    labs(title = label)
  a2 = ggplot(train.proyectados) + 
    geom_point(aes(x=X3, y=X4, 
                   group = grupo, color = grupo), 
               size=5, alpha = .2) + 
    geom_point(aes(x=X3, y=X4, 
                   group = grupo, color = grupo))+ 
    labs(title = label)
  list(a1,a2)
}
plot.ult <- function(V1, label, train.p, test.p){
  train.proyectados <- data.frame(as.matrix(train.p %>% 
                          dplyr::select(-grupo)) %*% V1) %>%
    mutate(grupo = train.p$grupo)
  
  test.proyectados <- data.frame(as.matrix(test.p %>% 
                          dplyr::select(-grupo)) %*% V1) %>%
    mutate(grupo = test.p$grupo)
  a7 =  ggplot(train.proyectados) + 
    geom_point(aes(x=X17, y=X18, 
                   group = grupo, color = grupo), 
               size=5, alpha = .2) + 
    geom_point(aes(x=X17, y=X18, 
                   group = grupo, color = grupo))+ 
    labs(title = label)
  
  a8 =  ggplot(train.proyectados) + 
    geom_point(aes(x=X19, y=X20, 
                   group = grupo, color = grupo), 
               size=5, alpha = .2) + 
    geom_point(aes(x=X19, y=X20, 
                   group = grupo, color = grupo))+ 
    labs(title = label)
  list(a7,a8)
}
prueba.20comp <- function(train.p, test.p, dataset){
  # Iterative LDA, projected to a 20-dimentional space
  modelo.iterativo <- lda.iter(train.p = train.p, 
                               test.p = test.p,
                               espacio.proy = 20)
  # Data.frame with information of (\rho and f(\rho))
  iter <- sapply(1:length(modelo.iterativo$iteraciones), 
                 function(x){
          data.frame(modelo.iterativo$iteraciones[[x]]$rho,
                modelo.iterativo$iteraciones[[x]]$frho)  %>% 
                     setNames(c("rho", 
                            "frho")) %>% data.frame()}) %>% 
    t() %>% data.frame() %>% 
    mutate(iter = 1:nrow(.), 
           rho = unlist(rho),
           frho = unlist(frho))
  iter <- iter %>% 
    dplyr::mutate(frho2 = c(iter$frho[2:nrow(.)],NA),
                  rho2 = c(iter$rho[2:nrow(.)], NA))
  
  # f(\rho) graph. It contains the iterations of the algorit
  plot <- iter %>% ggplot() +
    geom_point(aes(x = rho, y = frho))  + 
    geom_segment(aes(x = rho, y = frho, 
                     xend = rho, yend = frho2), 
                 color = "gray50") +
    geom_segment(aes(x = rho, y = frho2, 
                     xend = rho2, yend = frho2), 
                 arrow = arrow(length = unit(.3, "cm")), 
                 size = 1, 
                 color = "gray50") +
    labs(title = paste("Valor de la funcion f(rho) con 
      respecto al numero de iteraciones ", dataset,sep =""), 
         ylab = "f(rho)",
         xlab = "rho")
  # Save the image into a *.png file
  png(paste(getwd(), "/graphs/Chapter4_iteraciones_", 
      dataset, ".png", sep =""),width = 500, height = 500)
  grid.arrange(plot, ncol = 1)
  dev.off()
  
  # Projected data into a 20-dimensional space
  V <- modelo.iterativo$iteraciones[[
    length(modelo.iterativo$iteraciones)]]$V
  
  r1 <- 1
  r2 <- round(length(modelo.iterativo$iteraciones)/3*1)
  r3 <- round(length(modelo.iterativo$iteraciones))
  
  # iteration1
  V1 <- modelo.iterativo$iteraciones[[r1]]$V
  plot1 <- plot.comp(V1, paste("Iteracion ",r1, sep=""),
                     train.p, test.p)
  
  # iteracion3
  V2 <- modelo.iterativo$iteraciones[[r2]]$V
  plot2 <- plot.comp(V2, paste("Iteracion ",r2, sep=""),
                     train.p, test.p)
  
  # iteracion9
  V3 <- modelo.iterativo$iteraciones[[r3]]$V
  plot3 <- plot.comp(V3, paste("Iteracion ",r3, sep=""),
                     train.p, test.p)
  
  png(paste(getwd(),"/graphs/Chapter4_ejemplo20componentes_",
        dataset, ".png", sep =""),width = 700, height = 800)
  grid.arrange(plot1[[1]],plot1[[2]],plot2[[1]],
               plot2[[2]],plot3[[1]],plot3[[2]], 
               ncol = 2)
  dev.off()
  
  # last components
  V4 <- modelo.iterativo$iteraciones[[r3]]$V
  plot4 <- plot.ult(V4, "Ultimas componentes", 
                    train.p, test.p)
  png(paste(getwd(),"/graphs/Chapter4_ultimasComponentes_",
        dataset, ".png", sep =""),height = 400, width = 700)
  grid.arrange(plot4[[1]],plot4[[2]], ncol = 2)
  dev.off()
  # knn model ---------------------------------------------
  V <- modelo.iterativo$iteraciones[[
    length(modelo.iterativo$iteraciones)]]$V
  
  train.proyectados <- data.frame(as.matrix(train.p %>% 
                          dplyr::select(-grupo)) %*% V) %>%
    mutate(grupo = train.p$grupo)
  
  test.proyectados <- data.frame(as.matrix(test.p %>% 
                          dplyr::select(-grupo)) %*% V) %>%
    mutate(grupo = test.p$grupo)
  
  train.proyectados %>% head
  knn.model <- knn(train = train.proyectados[,c(1:20)], 
                   test = test.proyectados[, c(1:20)], 
                   cl = train.proyectados$grupo, k = 3)
  
  pp <- data.frame(original = test.proyectados$grupo, 
                   predicho = knn.model)
  
  pp <- pp %>% mutate(verdad = original != predicho)
  
  error <- sum(pp$verdad)/nrow(pp)
  list(modelo.iterativo=modelo.iterativo,iter = iter, V = V, 
       error = error, knn.model = knn.model)
}
\end{lstlisting}

\subsection{$functions\_profiling$}
\begin{lstlisting}[language=R, basicstyle=\small]
prolifiling.model <- function(train.p, test.p, 
                                dataset,numComp){

print(paste("====Realizando profiling de la base",dataset))
lda.time <- lapply(1:14, function(x){
  print(paste("Iteracion con: ", 
              numComp[x], " componentes."))
  
  lda.iter <- sapply(1:5, function(y){
    system.time(modelo.iterativo <- 
                  lda.iter.ef(train.p = train.p, 
                              test.p = test.p, 
                              espacio.proy = numComp[x]))
  }) %>% data.frame() %>% t() %>% data.frame() %>% 
    dplyr::select(elapsed) %>% 
    dplyr::summarise(mean = mean(elapsed))
  
  logistico <- sapply(1:5, function(y){
    system.time(modelo <- multinom(grupo~., 
          data = train.p[,c(1:numComp[x], dim(train.p)[2])], 
                                   MaxNWts = 10000))
  }) %>% data.frame() %>% t() %>% data.frame() %>% 
    dplyr::select(elapsed) %>% 
    dplyr::summarise(mean = mean(elapsed))
  
  lda.original <- sapply(1:5, function(y){
    system.time(modelo <- lda(grupo ~ .,
                train.p[,c(1:numComp[x], dim(train.p)[2])], 
                        prior = c(1,1,1,1,1,1,1,1,1,1)/10))
  })  %>% data.frame() %>% t() %>% data.frame() %>% 
    dplyr::select(elapsed) %>% 
    dplyr::summarise(mean = mean(elapsed))
  
  list(lda.iter, logistico, lda.original)
})

data.frame(unlist(lda.time)) %>% 
  mutate(var = rep(c("lda.iter", "logis", "lda.orig"), 14),
         ID = rep(1:(nrow(.)/3), each = 3)) %>% 
  setNames(c("time", "var", "ID")) %>% 
  spread(var, time)
}


graficar.profiling <- function(times.profiling, numComp,
                               dataset){
plot1 <- times.profiling %>% 
  mutate(numComp = numComp) %>% 
  gather(variable, value, lda.iter:logis) %>% 
  ggplot(aes(x = numComp , y = value,
             group = variable, color = variable)) +
  geom_point() + geom_line()+
  labs(title = "Tiempo de los modelos", 
       x = "Espacio de proyeccion", 
       y="Tiempo en (s)")
png(filename = paste(getwd(), "/graphs/Chapter4_profiling", 
      dataset, ".png", sep =""), width = 600, height = 400)
grid.arrange(plot1, ncol = 1)
dev.off()
}
\end{lstlisting}

\section{Preprocesamiento}

\subsection{$MNIST\_munge$}
\begin{lstlisting}[language=R, basicstyle=\small]
# 1) Load libraries ----------------------------------------
set.seed(12345)
library(plyr) # Tools for split, applying and combine data
library(dplyr) # package of grammar of data manipulation
library(tidyr) # package to efficiently "tidy" data
library(FactoMineR) # Package for multivariate analysis
library(ggplot2) # Graphing package "Grammar of Graphics"
library(gridExtra) # Package to work with grid graphics

# 1) Generate train and test from MNIST (run once) ---------
# darch::readMNIST("data/MNIST/")

# 2) Read MNIST data ---------------------------------------
# Original datasets
load("data/MNIST/train.RData")
load("data/MNIST/test.RData")

# Matrix with images in the rows and the variable in columns
numeros.ind.pix = rbind(data.frame(trainData, 
                  grupo = factor(apply(trainLabels[,], 1,  
                      function(x){ which(x == 1)-1 }))),
                  data.frame(testData, 
                      grupo = factor(apply(testLabels[,], 1,  
                      function(x){ which(x == 1)-1 })))) %>% 
  mutate(ID = 1:nrow(.))
base.mnist <- numeros.ind.pix
cache("base.mnist")
rm(testData, testLabels, trainData, trainLabels)

# 3) Train and test sets -----------------------------------
# Size of images per class
n.train <- 400

# Sample $n.train$ imager per class
train.id <- numeros.ind.pix %>% 
  group_by(grupo) %>% 
  sample_n(n.train) %>% 
  data.frame()

# Train and Test data.frames                    
train <- numeros.ind.pix[train.id$ID, ]
test <- numeros.ind.pix[-train.id$ID, ]

# Percentaje of images in train dataset
nrow(train)/(nrow(train)+nrow(test))

# 4) PCA ---------------------------------------------------
# PCA of MNIST dataset (each row is an image)
# 193 colums contains ~95% of the total variance
system.time(pca.train <- PCA(train %>% 
                               dplyr::select(-grupo, -ID), 
                             graph = F,
                             scale.unit = F,
                             ncp = 193))

# Matrix of loadings: 784 rows, 193 columns 
loadings <- sweep(pca.train$var$coord,2,  # 784 x 193
        sqrt(pca.train$eig[1:ncol(pca.train$var$coord),1]),
        FUN="/")

# Principal components 
train.p <- pca.train$ind$coord %>% data.frame() %>% 
  dplyr::mutate(grupo = train$grupo)

# [Revision] reconstuyendo primer componente 
as.matrix(scale(train %>% dplyr::select(-grupo, -ID), 
                center = apply(train %>% 
                      dplyr::select(-grupo, -ID), 2,mean), 
                scale = F)) %*% loadings[,1] %>% head

pca.train$ind$coord[,1] %>% head

# Projection of the test dataset with the train loadings 
# (centered with mean of train dataset)
system.time(test.p <- as.matrix(scale(test %>% 
            dplyr::select(-grupo, -ID), # 69,000 x 193
                                      
            center = apply(train %>% # centers of train
                    dplyr::select(-grupo, -ID), 2,mean), 
                          scale = F)) %*% loadings %>% 
              data.frame() %>% 
              mutate(grupo = test$grupo))
dim(test.p)
dim(train.p)
MNIST.test <- test.p
MNIST.train <- train.p
cache("MNIST.test")
cache("MNIST.train")
\end{lstlisting}

\subsection{$JAFFE\_munge$}
\begin{lstlisting}[language=R, basicstyle=\small]
# 1) Load libraries ----------------------------------------
set.seed(12345)
library(tiff) # package that reads *.tiff images 
library(tidyr) # package to efficiently "tidy" data
library(dplyr) # package of grammar of data manipulation
library(ProjectTemplate) # Template data analysis project
library(FactoMineR) # Package for multivariate analysis

# 2) Read databases ----------------------------------------
# Get the names of the files
files <- (Sys.glob("data/Jaffe/*.tiff"))

# Create an array that contains all the images paths 
nombres.personas <- apply(as.matrix(files),1,function(x){
  gsub("data/Jaffe/","",x)  
})

# Data.frame with info about each image (ID, name, class)
guia.personas.df <- data.frame(nombres.personas,
                               nombres.personas) %>%
  separate(nombres.personas.1, c("persona", 
                                 "postura", 
                                 "ID", 
                                 "extension")) %>%
  dplyr::select(-extension,
                -ID) %>%
  mutate(ID = 1:213, 
         nombres.personas = as.character(nombres.personas),
         persona = factor(persona),
         postura = factor(postura))

# List of length 213 that contains an image in each element
lista.completa.imagenes <- lapply(files, function(x){  
  imagen <- readTIFF(x)
  if(length(dim(imagen)) == 3){
    matrix <- as.vector(imagen[,,1])
  }else{
    matrix <- as.vector(imagen)
  }
})

# Data.frame, each column is an image
base.imagenes <- data.frame(lista.completa.imagenes)
names(base.imagenes) <- guia.personas.df$nombres.personas

# 2 matrices: In base.pix.ind.1, the columns are the images
#             In base.ind.pix.1, the rows are the images            
base.pix.ind.1 <- data.frame(base.imagenes)
base.ind.pix.1 <- data.frame(t(base.imagenes)) %>% 
  mutate(grupo = names(base.imagenes)) %>% 
  separate(grupo, c("grupo", 
                    "postura", 
                    "ID", 
                    "extension")) %>% 
  dplyr::select(-postura, 
                -ID, 
                -extension) %>% 
  mutate(ID = 1:nrow(.))

# 3) Explorar bases de datos -------------------------------
# sizes of the matrices
dim(base.pix.ind.1) # 65,536 rows and 213 columns
dim(base.ind.pix.1) # 213 rows and 65,536 columns

# save in cache the data.base
base.jaffe <- base.ind.pix.1
cache("base.jaffe")

# 4) Train and test  ---------------------------------------
# Size of images per class
n.train <- 5

# Sample $n.train$ imager per class
train.id <- base.jaffe %>% 
  group_by(grupo) %>% 
  sample_n(n.train) %>% 
  data.frame()

# Train and Test data.frames
train <- base.jaffe[train.id$ID, ]
test <- base.jaffe[-train.id$ID, ]

# Percentaje of images in train dataset
nrow(train)/(nrow(train)+nrow(test))

# 5) PCA ---------------------------------------------------

# PCA of JAFFE dataset (each row is an image)
# 52 colums contains ~95% of the total variance
system.time(pca.train <- PCA(train %>% 
                               dplyr::select(-grupo, -ID), 
                             graph = F,
                             scale.unit = F,
                             ncp = 49)) 

# Matrix of loadings: 65,536 rows, 49 columns
loadings <- sweep(pca.train$var$coord,2,
        sqrt(pca.train$eig[1:ncol(pca.train$var$coord),1]),
        FUN="/")

# Principal components 
train.p <- pca.train$ind$coord %>% data.frame() %>% 
  dplyr::mutate(grupo = train$grupo)

# [Revision] Reconstuyendo primer componente 
as.matrix(scale(train %>% dplyr::select(-grupo, -ID), 
                center = apply(train %>%
                        dplyr::select(-grupo, -ID), 2,mean), 
                scale = F)) %*% loadings[,1] %>% head

pca.train$ind$coord[,1] %>% head

# Projection of the test dataset with the train loadings 
# (centered with mean of train dataset)
system.time(test.p <- as.matrix(scale(test %>% 
                    dplyr::select(-grupo, -ID), 
                    center = apply(train %>% 
                       dplyr::select(-grupo, -ID), 2,mean), 
                    scale = F)) %*% loadings %>% 
              data.frame() %>% 
              mutate(grupo = test$grupo))

JAFFE.test <- test.p
JAFFE.train <- train.p
cache("JAFFE.test")
cache("JAFFE.train")

rm(list = ls())
\end{lstlisting}



